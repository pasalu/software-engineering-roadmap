% You should title the file with a .tex extension (hw1.tex, for example)
\documentclass[11pt]{article}

\usepackage{amsmath}
\usepackage{amssymb}
\usepackage{fancyhdr}
\usepackage{listings}
\usepackage{color}
\usepackage{graphicx}
\graphicspath{ {images/} }
\usepackage{hyperref}
\usepackage{mathtools}

\definecolor{dkgreen}{rgb}{0,0.6,0}
\definecolor{gray}{rgb}{0.5,0.5,0.5}
\definecolor{mauve}{rgb}{0.58,0,0.82}

\lstset{frame=tb,
  language=Java,
  aboveskip=3mm,
  belowskip=3mm,
  showstringspaces=false,
  columns=flexible,
  basicstyle={\small\ttfamily},
  numbers=none,
  numberstyle=\tiny\color{gray},
  keywordstyle=\color{blue},
  commentstyle=\color{dkgreen},
  stringstyle=\color{mauve},
  breaklines=true,
  breakatwhitespace=true,
  tabsize=3
}

\oddsidemargin0cm
\topmargin-2cm     %I recommend adding these three lines to increase the 
\textwidth16.5cm   %amount of usable space on the page (and save trees)
\textheight23.5cm  

\newcommand{\question}[2] {\vspace{.25in} \hrule\vspace{0.5em}
\noindent{\bf #1: #2} \vspace{0.5em}
\hrule \vspace{.10in}}
\renewcommand{\part}[1] {\vspace{.10in} {\bf (#1)}}

\newcommand{\myname}{Chang-Hyun Mungai}
\newcommand{\myhwnum}{Bit Manipulation Notes}

\setlength{\parindent}{0pt}
\setlength{\parskip}{5pt plus 1pt}
 
\pagestyle{fancyplain}
\lhead{\fancyplain{}{\textbf{\myhwnum}}}      % Note the different brackets!
\rhead{\fancyplain{}{\myname\\ \myandrew}}

\begin{document}

\medskip                        % Skip a "medium" amount of space
                                % (latex determines what medium is)
                                % Also try: \bigskip, \littleskip

\thispagestyle{plain}
\begin{center}                  % Center the following lines
{\Large Bit Manipulation} \\
\end{center}


\question{Basics}

\begin{table}[h!]
\begin{center}
\begin{tabular}{l|r|r|r|r}
And (\&): & 0 \& 0 = 0 & 1 \& 0 = 0 & 0 \& 1 = 0 & 1 \& 1 = 1\\
\hline
Or (\textbar): & 0 \textbar 0 = 0 & 1 \textbar 0 = 1 & 0 \textbar 1 = 1 & 1 \textbar 1 = 1\\
\hline
Xor ($\bigwedge$): & 0 \textasciicircum 0 = 0 & 1 \textasciicircum 0 = 0 & 0 \textasciicircum 1 = 0 & 1 \textasciicircum 1 = 1\\
\end{tabular}
\end{center}
\end{table}

\question{Shifts}

\begin{itemize}
  \item Left Shift: If you run out of space the bits drop off
\begin{itemize}
  \item Both arith and logical shift in 0
\end{itemize}
  \begin{enumerate}
    \item 00011001 \textless \textless 2 = 01100100
    \item 00011001 \textless \textless 4 = 10010000
  \end{enumerate}
  \item Right Shift if you run out of space the bits drop off
\begin{itemize}
  \item arithmetic shift - shift in sign bit (sticky shift), logical-shift in 0
\end{itemize}
  \begin{enumerate}
  \item 00011001 \textgreater \textgreater 2 = 00000110
  \item 00011001 \textgreater \textgreater 4 = 00000001
  \end{enumerate}
\end{itemize}

\question{Notes}

\begin{itemize}
  \item Windows calculator can do operations in binary, view programmer
\end{itemize}

\end{document}

