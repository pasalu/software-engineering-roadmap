% You should title the file with a .tex extension (hw1.tex, for example)
\documentclass[11pt]{article}

\usepackage{amsmath}
\usepackage{amssymb}
\usepackage{fancyhdr}
\usepackage{listings}
\usepackage{color}
\usepackage{graphicx}
\graphicspath{ {images/} }
\usepackage{hyperref}
\usepackage{mathtools}

\definecolor{dkgreen}{rgb}{0,0.6,0}
\definecolor{gray}{rgb}{0.5,0.5,0.5}
\definecolor{mauve}{rgb}{0.58,0,0.82}

\lstset{frame=tb,
  language=Java,
  aboveskip=3mm,
  belowskip=3mm,
  showstringspaces=false,
  columns=flexible,
  basicstyle={\small\ttfamily},
  numbers=none,
  numberstyle=\tiny\color{gray},
  keywordstyle=\color{blue},
  commentstyle=\color{dkgreen},
  stringstyle=\color{mauve},
  breaklines=true,
  breakatwhitespace=true,
  tabsize=3
}

\oddsidemargin0cm
\topmargin-2cm     %I recommend adding these three lines to increase the 
\textwidth16.5cm   %amount of usable space on the page (and save trees)
\textheight23.5cm  

\newcommand{\question}[2] {\vspace{.25in} \hrule\vspace{0.5em}
\noindent{\bf #1: #2} \vspace{0.5em}
\hrule \vspace{.10in}}
\renewcommand{\part}[1] {\vspace{.10in} {\bf (#1)}}

\newcommand{\myname}{Chang-Hyun Mungai}
\newcommand{\myhwnum}{Design for Robustness notes}

\setlength{\parindent}{0pt}
\setlength{\parskip}{5pt plus 1pt}
 
\pagestyle{fancyplain}
\lhead{\fancyplain{}{\textbf{\myhwnum}}}      % Note the different brackets!

\begin{document}

\medskip                        % Skip a "medium" amount of space
                                % (latex determines what medium is)
                                % Also try: \bigskip, \littleskip

\thispagestyle{plain}
\begin{center}                  % Center the following lines
{\Large Design for Robustness} \\
\end{center}

\question{Exceptions}

\begin{itemize}
  \item Notify caller of exceptional circumstance (usually operational failure)
  \item Semantics
\begin{itemize}
  \item An exception propagates up the function  \item call stack until main() is reached (terminates program) or until the exception is caught
\end{itemize}
  \item Sources of exceptions:
\begin{itemize}
  \item Program throwing an exception
  \item Exception thrown by the Java Virtual Machine
\end{itemize}
\end{itemize}

\question{Java: Finally}

The finally block always runs after try/catch

\question{Design choice: checked and unchecked}

\begin{itemize}
  \item Unchecked exception: any subclass of RuntimeException
\begin{itemize}
  \item Error which is highly unlikely and/or unrecoverable
\end{itemize}
  \item check exception: any subclass of Exception that is not a subclass of RuntimeException
\begin{itemize}
  \item Error that every caller should be aware of and explicitly decide to handle or pass on
\end{itemize}
  \item Return values(  - 1, false, null): If failure is common and expected possibility
\end{itemize}

\question{Creating and throwing your own exceptions}

\begin{itemize}
  \item Methods must declare any checked exceptions they might throw
  \item If your class extends java.lang.Throwable you can throw it:
\begin{itemize}
  \item
\begin{lstlisting}
 if (some__){
	throw new customException("Blah blah");
  }
\end{lstlisting}
\end{itemize}
\end{itemize}

\question{Benefits of exceptions}

\begin{itemize}
  \item High level summary of error and stack trace
\begin{itemize}
  \item Compare: core dumped in C
\end{itemize}
  \item Can't forget to handle common failure modes
\begin{itemize}
  \item Compare: using a flag or special return value
\end{itemize}
  \item Can optionally recover from failure
\begin{itemize}
  \item Compare: calling System.exit()
\end{itemize}
  \item Improve code structure
\begin{itemize}
  \item   \item Separate routine operations from error  \item handling
\end{itemize}
  \item Allow consisten clean  \item up in both normal and exceptional operation
\end{itemize}

\question{Guide for exceptions}

\begin{itemize}
  \item Catch and handle all checked exceptions
\begin{itemize}
  \item Unless there is no good way
\end{itemize}
  \item Use runtime exceptions for programming error
  \item Other
\begin{itemize}
  \item Don't catch an exception without (at least somewhat) handling the error
  \item When you throw an exception describe the error
  \item if you re  \item throw an exception, always include the original exception as the clause
\end{itemize}
\end{itemize}

\question{Modular Protection}

\begin{itemize}
  \item Errors and bugs unavoidable but exceptions should not leak across modules (methods, classes) if possible
  \item Good modules handles exceptional conditions locally
\begin{itemize}
  \item Local input validation and local exception handling where possible
  \item Explicit interfaces with clear pre/post conditions
\end{itemize}
  \item Explicitly documented and checked exceptions where exceptional conditions may propagate between modules
\begin{itemize}
  \item Information hiding -encapsulation of critical code (likely bugs, likely exceptions)
\end{itemize}
\end{itemize}

\question{Problems when testing (sub - )systems}

\begin{itemize}
  \item User interfaces and user interactions
\begin{itemize}
  \item Users click buttons, interpret output
  \item Waiting/timing issues
\end{itemize}
  \item Test data vs. real data
  \item Testing against big infrastructure (database, web services,..)
  \item Testing with side effects (e.g. printing and mailing documents)
  \item Nondeterministic behavior
  \item Concurrency
\end{itemize}

\question{Testing strategies in environments}

\begin{itemize}
  \item Separate business logic and data representation from GUI for testing
  \item Test algorithms locally without large environment using stubs
  \item Advantage of stubs
\begin{itemize}
  \item Create deterministic response
  \item Can reliably simulate spurious states (network error)
  \item Can speed up test execution
  \item Can simulate functionality not yet implemented
\end{itemize}
  \item Automate
\end{itemize}

\question{Scaffolding}

\begin{itemize}
  \item Catch bugs early: Before client code or service available
  \item Limit scope of debugging: Localize errors
  \item Improve coverage
\begin{itemize}
  \item System level tests may only cover 70% of code
  \item Simulate unusual error conditions
\end{itemize}
  \item Validate internal interface/API designs
\begin{itemize}
  \item Simulate clients in advance of their developement
  \item Simulate services in advance of their development
\end{itemize}
  \item Capture developer intent (in absence of specification documentation)
\begin{itemize}
  \item A test suite formally captures elements of design intent
  \item Developer documentation
\end{itemize}
  \item Improve low  \item level design
\begin{itemize}
  \item Early attention to ability to test
\end{itemize}
\end{itemize}

\end{document}

