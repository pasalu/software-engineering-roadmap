% You should title the file with a .tex extension (hw1.tex, for example)
\documentclass[11pt]{article}

\usepackage{amsmath}
\usepackage{amssymb}
\usepackage{fancyhdr}
\usepackage{listings}
\usepackage{color}
\usepackage{graphicx}
\graphicspath{ {images/} }
\usepackage{hyperref}
\usepackage{mathtools}

\definecolor{dkgreen}{rgb}{0,0.6,0}
\definecolor{gray}{rgb}{0.5,0.5,0.5}
\definecolor{mauve}{rgb}{0.58,0,0.82}

\lstset{frame=tb,
  language=Java,
  aboveskip=3mm,
  belowskip=3mm,
  showstringspaces=false,
  columns=flexible,
  basicstyle={\small\ttfamily},
  numbers=none,
  numberstyle=\tiny\color{gray},
  keywordstyle=\color{blue},
  commentstyle=\color{dkgreen},
  stringstyle=\color{mauve},
  breaklines=true,
  breakatwhitespace=true,
  tabsize=3
}

\oddsidemargin0cm
\topmargin-2cm     %I recommend adding these three lines to increase the 
\textwidth16.5cm   %amount of usable space on the page (and save trees)
\textheight23.5cm  

\newcommand{\question}[2] {\vspace{.25in} \hrule\vspace{0.5em}
\noindent{\bf #1: #2} \vspace{0.5em}
\hrule \vspace{.10in}}
\renewcommand{\part}[1] {\vspace{.10in} {\bf (#1)}}

\newcommand{\myname}{Chang-Hyun Mungai}
\newcommand{\myhwnum}{Software Design Notes}

\setlength{\parindent}{0pt}
\setlength{\parskip}{5pt plus 1pt}
 
\pagestyle{fancyplain}
\lhead{\fancyplain{}{\textbf{\myhwnum}}}      % Note the different brackets!

\begin{document}

\medskip                        % Skip a "medium" amount of space
                                % (latex determines what medium is)
                                % Also try: \bigskip, \littleskip

\thispagestyle{plain}
\begin{center}                  % Center the following lines
{\Large Software Design} \\
\end{center}

\question{Software design and complexity}
	
\begin{itemize}
\item scale
\item environment (I/O, Network)
\item infrastructure (libraries, frameworks)
\item evolution (design for change)
\item correctness (testing, analysis tools, automation)
\end{itemize}

\question{software qualities}

\begin{itemize}
\item sufficiency/func correctness
\item robustness
\item flexibility
\item reusability
\item efficiency
\item scalability
\item security
\end{itemize}

\question{A simple process}

\begin{itemize}
\item Discuss the software that needs to be written
\item Write some code
\item Test the code to identify the defects
\item Debug to find causes of defects
\item Fix the defects
\item If not done, return to first step
\end{itemize}

\question{design tips}

\begin{itemize}
\item Think before coding
\item Consider quality attributes (maintainability,extensibility, performance)
\item Consider alternatives and make conscious design decisions
\end{itemize}

\question{Preview: The design process}

\begin{itemize}
\item Object‐Oriented Analysis
\begin{itemize}
\item Understand the problem
\item Identify the key concepts and their relationships
\item Build a (visual) vocabulary
\item Create a domain model (aka conceptual model)
\end{itemize}
\item Object‐Oriented Design
\begin{itemize}
\item Identify software classes and their relationships with class diagrams
\item Assign responsibilities (attributes, methods)
\item Explore behavior with interaction diagrams
\item Explore design alternatives
\item Create an object model (aka design model and design class diagram) and interaction models
\end{itemize}
\item Implementation
\begin{itemize}
\item Map designs to code, implementing classes and methods
\end{itemize}
\end{itemize}

\question{Objects}

\begin{itemize}
\item A package of state (data) and behavior (actions)
\item Can interact with objects by sending messages
\begin{itemize}
\item perform an action (e.g., move)
\item request some information (e.g., getSize)
\end{itemize}
\item Possible messages described through an interface
\end{itemize}

\question{sources}

\begin{itemize}
\item \url{http://www.cs.cmu.edu/~charlie/courses/15-214/2015-fall/index.html#schedule}
\end{itemize}

\end{document}

