% You should title the file with a .tex extension (hw1.tex, for example)
\documentclass[11pt]{article}

\usepackage{amsmath}
\usepackage{amssymb}
\usepackage{fancyhdr}
\usepackage{listings}
\usepackage{color}
\usepackage{graphicx}
\graphicspath{ {images/} }
\usepackage{hyperref}
\usepackage{mathtools}

\definecolor{dkgreen}{rgb}{0,0.6,0}
\definecolor{gray}{rgb}{0.5,0.5,0.5}
\definecolor{mauve}{rgb}{0.58,0,0.82}

\lstset{frame=tb,
  language=Java,
  aboveskip=3mm,
  belowskip=3mm,
  showstringspaces=false,
  columns=flexible,
  basicstyle={\small\ttfamily},
  numbers=none,
  numberstyle=\tiny\color{gray},
  keywordstyle=\color{blue},
  commentstyle=\color{dkgreen},
  stringstyle=\color{mauve},
  breaklines=true,
  breakatwhitespace=true,
  tabsize=3
}

\oddsidemargin0cm
\topmargin-2cm     %I recommend adding these three lines to increase the 
\textwidth16.5cm   %amount of usable space on the page (and save trees)
\textheight23.5cm  

\newcommand{\question}[2] {\vspace{.25in} \hrule\vspace{0.5em}
\noindent{\bf #1: #2} \vspace{0.5em}
\hrule \vspace{.10in}}
\renewcommand{\part}[1] {\vspace{.10in} {\bf (#1)}}

\newcommand{\myname}{Chang-Hyun Mungai}
\newcommand{\myhwnum}{Operating System notes}

\setlength{\parindent}{0pt}
\setlength{\parskip}{5pt plus 1pt}
 
\pagestyle{fancyplain}
\lhead{\fancyplain{}{\textbf{\myhwnum}}}      % Note the different brackets!

\begin{document}

\medskip                        % Skip a "medium" amount of space
                                % (latex determines what medium is)
                                % Also try: \bigskip, \littleskip

\thispagestyle{plain}
\begin{center}                  % Center the following lines
{\Large Operating System} \\
\end{center}

\question{Processes}

\begin{itemize}
  \item instance of computer program
  \item abstraction for a running program
  \item single CPU can appear to execute multiple processes concurrently by having the processor switch among them
  \item OS performs this using contect switching
  \item OS keeps track all state information for a process (context)
  \begin{itemize}
    \item values of PC, register file, main memory
  \end{itemize}
  \item single processor can only execute code for single process
  \item when OS decided to transfer control to new process performs context switch
  \begin{itemize}
    \item save contect of current process, restore contect of new process, pass control to new context
  \end{itemize}
  \item system call: special function, passes control to OS and saves shell context, creates new context (hello) and gives it control
  \item scheduling: kernel decides to preempt the current process and restart a previously preempted process
  \begin{itemize}
    \item handled by code in the kernel called scheduler
  \end{itemize}
  \item sum: decision: scheduling, act: context switch
\end{itemize}

\question{Threads}

\begin{itemize}
  \item execution unit
  \item process can be made up of multiple threads that execute concurrently on process context with same code and global data
  \item requirement for concurrency in network servers make threads important
  \begin{itemize}
    \item easier to share data between multiple threads than multiple processes
    \item threads more efficient than processes
  \end{itemize}
  \item multithreaing a method to make programs faster with multiple processors
\end{itemize}

\question{Virtual Memory}

\begin{itemize}
  \item abstraction provides each process the illusion that it has exclusive use of the main memory
  \begin{itemize}
    \item heap: follows code and data areas (which are themselves fixed in size)
    \begin{itemize}
      \item expands and contracts dynamically at runtime (malloc, free)
    \end{itemize}
    \item Shared libraies: near the middle
    \item stack: at the top
    \begin{itemize}
      \item used to implement function calls
      \item expands and shrinks dynamically (grows when you call/shrinks when you return from a function)
    \end{itemize}
    \item kernel virtual memory
  \end{itemize}
\end{itemize}

\question{sources}

\begin{itemize}
\item Computer Systems: A Programmers Perspective
\end{itemize}

\end{document}

