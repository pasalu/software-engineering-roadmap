% You should title the file with a .tex extension (hw1.tex, for example)
\documentclass[11pt]{article}

\usepackage{amsmath}
\usepackage{amssymb}
\usepackage{fancyhdr}
\usepackage{listings}
\usepackage{color}
\usepackage{graphicx}
\graphicspath{ {images/} }
\usepackage{hyperref}
\usepackage{mathtools}

\definecolor{dkgreen}{rgb}{0,0.6,0}
\definecolor{gray}{rgb}{0.5,0.5,0.5}
\definecolor{mauve}{rgb}{0.58,0,0.82}

\lstset{frame=tb,
  language=Java,
  aboveskip=3mm,
  belowskip=3mm,
  showstringspaces=false,
  columns=flexible,
  basicstyle={\small\ttfamily},
  numbers=none,
  numberstyle=\tiny\color{gray},
  keywordstyle=\color{blue},
  commentstyle=\color{dkgreen},
  stringstyle=\color{mauve},
  breaklines=true,
  breakatwhitespace=true,
  tabsize=3
}

\oddsidemargin0cm
\topmargin-2cm     %I recommend adding these three lines to increase the 
\textwidth16.5cm   %amount of usable space on the page (and save trees)
\textheight23.5cm  

\newcommand{\question}[2] {\vspace{.25in} \hrule\vspace{0.5em}
\noindent{\bf #1: #2} \vspace{0.5em}
\hrule \vspace{.10in}}
\renewcommand{\part}[1] {\vspace{.10in} {\bf (#1)}}

\newcommand{\myname}{Chang-Hyun Mungai}
\newcommand{\myhwnum}{Hardware Systems notes}

\setlength{\parindent}{0pt}
\setlength{\parskip}{5pt plus 1pt}
 
\pagestyle{fancyplain}
\lhead{\fancyplain{}{\textbf{\myhwnum}}}      % Note the different brackets!

\begin{document}

\medskip                        % Skip a "medium" amount of space
                                % (latex determines what medium is)
                                % Also try: \bigskip, \littleskip

\thispagestyle{plain}
\begin{center}                  % Center the following lines
{\Large Hardware Systems} \\
\end{center}

\question{Hardware Organization}

\begin{itemize}
  \item buses
  \begin{itemize}
    \item carry bytes of info between componenets
    \item words: fixed sized chunks of bytes (4 bytes/32 bits or 8 bytes/64 bits)
  \end{itemize}
  \item I/O devices
  \begin{itemize}
    \item system's connection to outside world by controller or adapter
    \item display, disk, mouse
    \item controller: chip sets in device itself or motherboard
    \item adapter plugs into slots on motherboard
  \end{itemize}
  \item Main Memory
  \begin{itemize}
    \item temporary storage for both program and data it manipulates
    \item physically dynamic random access memory (DRAM)
    \item logically organized as linear array of bytes each with its own unique address
  \end{itemize}
  \item Processor
  \begin{itemize}
    \item central processing unit (cpu)
    \item interprets (or executes) instructions stored in main memory
    \item at its core is a word sized register called program counter (PC)
      \begin{itemize}
      \item At any point points at machine language instruction in main memory
      \end{itemize}
    \item executes instruction and updates pc to next instruction
    \item instructions revolve around main memory, register file, and arithmetic/logic unit
      \begin{itemize}
      \item reigter file small storage device with word sized registers each with its own unique name
      \item ALU computers new data and address values
      \end{itemize}
    \item Simple operations CPU carries out at the request of an instruction
      \begin{itemize}
      \item Load: Copy a byte or a word from main memory into a register, overwriting the previous contents
      \item Store: Copy a byte or a word from a register to a location in main memory, overwriting the previous contents
      \item Operate: Copy the contents of two registers to the ALU, perform an arithmetic operation on the two words, and store the result in a register, overwriting the previous contents
      \item Jump: Extract a word from the instruction, copy that word into the program counter (PC), overwriting the previous value
    \end{itemize}
  \end{itemize}
\end{itemize}

\question{sources}

\begin{itemize}
\item Computer Systems: A Programmers Perspective pg 7-10
\end{itemize}

\end{document}

