% You should title the file with a .tex extension (hw1.tex, for example)
\documentclass[11pt]{article}

\usepackage{amsmath}
\usepackage{amssymb}
\usepackage{fancyhdr}
\usepackage{listings}
\usepackage{color}
\usepackage{graphicx}
\graphicspath{ {images/} }
\usepackage{hyperref}
\usepackage{mathtools}

\definecolor{dkgreen}{rgb}{0,0.6,0}
\definecolor{gray}{rgb}{0.5,0.5,0.5}
\definecolor{mauve}{rgb}{0.58,0,0.82}

\lstset{frame=tb,
  language=Java,
  aboveskip=3mm,
  belowskip=3mm,
  showstringspaces=false,
  columns=flexible,
  basicstyle={\small\ttfamily},
  numbers=none,
  numberstyle=\tiny\color{gray},
  keywordstyle=\color{blue},
  commentstyle=\color{dkgreen},
  stringstyle=\color{mauve},
  breaklines=true,
  breakatwhitespace=true,
  tabsize=3
}

\oddsidemargin0cm
\topmargin-2cm     %I recommend adding these three lines to increase the 
\textwidth16.5cm   %amount of usable space on the page (and save trees)
\textheight23.5cm  

\newcommand{\question}[2] {\vspace{.25in} \hrule\vspace{0.5em}
\noindent{\bf #1: #2} \vspace{0.5em}
\hrule \vspace{.10in}}
\renewcommand{\part}[1] {\vspace{.10in} {\bf (#1)}}

\newcommand{\myname}{Chang-Hyun Mungai}
\newcommand{\myhwnum}{Linked Lists Notes}

\setlength{\parindent}{0pt}
\setlength{\parskip}{5pt plus 1pt}
 
\pagestyle{fancyplain}
\lhead{\fancyplain{}{\textbf{\myhwnum}}}      % Note the different brackets!

\begin{document}

\medskip                        % Skip a "medium" amount of space
                                % (latex determines what medium is)
                                % Also try: \bigskip, \littleskip

\thispagestyle{plain}
\begin{center}                  % Center the following lines
{\Large Linked Lists} \\
\end{center}

\question{Big O}

\begin{itemize}
  \item space O(n)
  \item time
\begin{itemize}
  \item search worst O(n), average O(n)
  \item insert worst O(1), average O(1)
  \item delete worst O(1), average O(1)
\end{itemize}
\end{itemize}

\question{Advantages}

\begin{itemize}
  \item linked lists are a dynamic data structure, allocating the needed memory while the program is running
  \item insertion and deletion node operations are easily implemented in a linked list
  \item linear data structures such as stacks and queues are easily executed with a linked list
  \item they can reduce access time and may expand in real time without memory overhead
\end{itemize}

\question{Disadvantages}

\begin{itemize}
  \item they have a tendency to use more memory due to pointers requiring extra storage space
  \item nodes in a linked list must be read in order from the beginning as linked lists are inherently sequential access
  \item nodes are stored incontiguously, greatly increasing the time required to access individual elements within the list
  \item difficulties arise in linked lists when it comes to reverse traversing
  \begin{itemize}
      \item singly linked lists are cumbersome to navigate backwards[1] and while doubly linked lists are somewhat easier to read, memory is wasted in allocating space for a back pointer
\end{itemize}
\end{itemize}

\question{Uses}

\begin{itemize}
  \item stack
  \item queue
  \item memory allocation
\end{itemize}

\question{Creating a Linked List}

\begin{lstlisting}
class Node {
	Node next = null;
	int data;
	public Node(int d) { data = d; }
	void appendToTail(int d) {
		Node end = new Node(d);
		Node n = this;
		while (n.next != null) { n = n.next; }
		n.next = end;
	}
}
\end{lstlisting}

\question{Deleting a node}

\begin{lstlisting}
Node deleteNode(Node head, int d) {
	Node n = head;
	if (n.data == d) {
		return head.next;  /* moved head */
	}
	while (n.next != null) {
		if (n.next.data == d) {
			n.next = n.next.next;
			/* head didnt change */
			return head;
		}
		n = n.next;
	}
}
\end{lstlisting}

\question{Notes}

\begin{itemize}
  \item alternative to array to implement stack and queue
  \item allows any length
\end{itemize}

\end{document}

