% You should title the file with a .tex extension (hw1.tex, for example)
\documentclass[11pt]{article}

\usepackage{amsmath}
\usepackage{amssymb}
\usepackage{fancyhdr}
\usepackage{listings}
\usepackage{color}
\usepackage{graphicx}
\graphicspath{ {images/} }
\usepackage{hyperref}
\usepackage{mathtools}

\definecolor{dkgreen}{rgb}{0,0.6,0}
\definecolor{gray}{rgb}{0.5,0.5,0.5}
\definecolor{mauve}{rgb}{0.58,0,0.82}

\lstset{frame=tb,
  language=Java,
  aboveskip=3mm,
  belowskip=3mm,
  showstringspaces=false,
  columns=flexible,
  basicstyle={\small\ttfamily},
  numbers=none,
  numberstyle=\tiny\color{gray},
  keywordstyle=\color{blue},
  commentstyle=\color{dkgreen},
  stringstyle=\color{mauve},
  breaklines=true,
  breakatwhitespace=true,
  tabsize=3
}

\oddsidemargin0cm
\topmargin-2cm     %I recommend adding these three lines to increase the 
\textwidth16.5cm   %amount of usable space on the page (and save trees)
\textheight23.5cm  

\newcommand{\question}[2] {\vspace{.25in} \hrule\vspace{0.5em}
\noindent{\bf #1: #2} \vspace{0.5em}
\hrule \vspace{.10in}}
\renewcommand{\part}[1] {\vspace{.10in} {\bf (#1)}}

\newcommand{\myname}{Chang-Hyun Mungai}
\newcommand{\myhwnum}{Java Nuances Notes}

\setlength{\parindent}{0pt}
\setlength{\parskip}{5pt plus 1pt}
 
\pagestyle{fancyplain}
\lhead{\fancyplain{}{\textbf{\myhwnum}}}      % Note the different brackets!

\begin{document}

\medskip                        % Skip a "medium" amount of space
                                % (latex determines what medium is)
                                % Also try: \bigskip, \littleskip

\thispagestyle{plain}
\begin{center}                  % Center the following lines
{\Large Java Nuances} \\
\end{center}

\question{General Tips}

\begin{itemize}
  \item Getter and setter  
  \item Override and super
  \item Java outmatically collects garbage  
  \item \&\&/$\mid\mid$ checks left first
  \item + strings makes a new string every time, if you want to do in a loop use stringbuilder(reduce memory)
  \item Everything in Java not explicitly set to something, is initialized to a zero value

  \begin{itemize}
    \item references (anything that holds an object):null
    \item int/short/byte:0
    \item float/double:0.0
    \item booleans: false.
    \item array of something, all entries are also zeroed
  \end{itemize}

\end{itemize}

\question{Useful built in functions}

\begin{itemize}
  \item Arrays
\begin{itemize}
  \item Arrays.binarySearch(arr, target)
  \begin{itemize}
    \item Negative value shows where it should be
  \end{itemize}
  \item Arrays.sort(arr)
\end{itemize}
\end{itemize}

\question{Switch Statement}

\begin{itemize}
  \item All matching cases will be run unless their is a break statement
  \item Example
  \begin{itemize}
  \item 
    \begin{lstlisting}
switch (month) {
            case 1:  monthString = "January";
                     break;
            case 2:  monthString = "February";
                     break;
            case 3:  monthString = "March";
                     break;
            case 4:  monthString = "April";
                     break;
            case 5:  monthString = "May";
                     break;
            case 6:  monthString = "June";
                     break;
            case 7:  monthString = "July";
                     break;
            case 8:  monthString = "August";
                     break;
            case 9:  monthString = "September";
                     break;
            case 10: monthString = "October";
                     break;
            case 11: monthString = "November";
                     break;
            case 12: monthString = "December";
                     break;
            default: monthString = "Invalid month";
                     break;
        }
    \end{lstlisting}
  \end{itemize}
\end{itemize}

\question{Breaking out of for loops}

\begin{itemize}
  \item if you want to skip a particular iteration, use continue
  \begin{itemize}
    \item

\begin{lstlisting}
for(int i=0 ; i<5 ; i++){

    if (i==2){

      continue;
    }
}
\end{lstlisting}

\end{itemize}
  \item if you want to break out of the immediate loop use break
  \begin{itemize}
    \item

\begin{lstlisting}
for(int i=0 ; i<5 ; i++){

    if (i==2){

        break;
    }
}
\end{lstlisting}

  \end{itemize}
  \item if there are 2 loop, outer and inner.... and you want to break out of both the loop from  the inner loop, use break with label
  \begin{itemize}
    \item

\begin{lstlisting}
lab1: for(int j=0 ; j<5 ; j++){
     for(int i=0 ; i<5 ; i++){

        if (i==2){

          break lab1;
        }
     }
}
\end{lstlisting}

  \end{itemize}
\end{itemize}

\question{Generics}

\begin{itemize}
  \item Definition
  \begin{itemize}
    \item generics are a facility of generic programming
    \begin{itemize}
       \item a style of computer programming in which algorithms are written in terms of types to-be-specified-later that are then instantiated when needed for specific types provided as parameters
    \end{itemize}
    \item ex: compiletime: List<String> runtime:List
  \end{itemize}
  \item Notes
  \begin{itemize}
    \item in java, generics are only checked at compile time for type correctness
    \item generic type information is then removed via a process called type erasure, to maintain compatibility with old JVM implementations, making it unavailable at runtime
  \end{itemize}
  \item Sources
  \begin{itemize}
    \item \url{https://en.wikipedia.org/wiki/Generics_in_Java}
  \end{itemize}
\end{itemize}

\question{Type Classifications}

\begin{itemize}
  \item Concrete Types
  \begin{itemize}
    \item concrete types describe object implementations, including memory layout and the code executed upon method invocation
    \item the exact class of which an object is an instance not the more general set of the class and its subclasses
    \item beware of falling into the trap of thinking that all concrete types are single classes!
    \item Set of Exact Classes
    \item ex: List x has concrete type {ArrayList, LinkedList, ...}
  \end{itemize}
  \item Abstract Types
  \begin{itemize}
    \item Abstract types, on the other hand, describe properties of objects
    \item They do not distinguish between different implementations of the same behavior
    \item Java provides abstract types in the form of interfaces, which list the fields and operations that implementations must support
  \end{itemize}
\end{itemize}

\question{Access Modifiers}

\begin{itemize}
  \item public

  \begin{itemize}
    \item any class can access
    \item accessible by entire application
  \end{itemize}

  \item private

  \begin{itemize}
    \item only accessible within the class
  \end{itemize}

  \item protected

  \begin{itemize}
    \item allow the class itself to access them
    \item classes inside of the same package to access them
    \item subclasses of that class to access them
  \end{itemize}

  \item package protected

  \begin{itemize}
    \item default
    \item the same class and any class in the same package has access
    \item protected minus the subclass unless subclass is in package
  \end{itemize}

  \item Static: Belongs to class not an instance of the class
\end{itemize}

\question{Things to override in new object (for hashing and equality uses)}

  \begin{itemize}
    \item public int hashCode()
    \item public boolean equals(Object object)
  \end{itemize}

\begin{lstlisting}
ex: Tiger
	@Override
	public boolean equals(Object object) {
		boolean result = false;
		if (object == null || object.getClass() != getClass()) {
			result = false;
		} else {
			Tiger tiger = (Tiger) object;
			if (this.color == tiger.getColor()
					&& this.stripePattern == tiger.getStripePattern()) {
				result = true;
			}
		}
		return result;
	}
\end{lstlisting}

\question{Sources}

\begin{itemize}
\item \url{https://www.cs.utexas.edu/~scottm/cs307/codingSamples.htm}
\end{itemize}

\end{document}

