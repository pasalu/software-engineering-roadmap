% You should title the file with a .tex extension (hw1.tex, for example)
\documentclass[11pt]{article}

\usepackage{amsmath}
\usepackage{amssymb}
\usepackage{fancyhdr}
\usepackage{listings}
\usepackage{color}
\usepackage{graphicx}
\graphicspath{ {images/} }
\usepackage{hyperref}
\usepackage{mathtools}

\definecolor{dkgreen}{rgb}{0,0.6,0}
\definecolor{gray}{rgb}{0.5,0.5,0.5}
\definecolor{mauve}{rgb}{0.58,0,0.82}

\lstset{frame=tb,
  language=Java,
  aboveskip=3mm,
  belowskip=3mm,
  showstringspaces=false,
  columns=flexible,
  basicstyle={\small\ttfamily},
  numbers=none,
  numberstyle=\tiny\color{gray},
  keywordstyle=\color{blue},
  commentstyle=\color{dkgreen},
  stringstyle=\color{mauve},
  breaklines=true,
  breakatwhitespace=true,
  tabsize=3
}

\oddsidemargin0cm
\topmargin-2cm     %I recommend adding these three lines to increase the 
\textwidth16.5cm   %amount of usable space on the page (and save trees)
\textheight23.5cm  

\newcommand{\question}[2] {\vspace{.25in} \hrule\vspace{0.5em}
\noindent{\bf #1: #2} \vspace{0.5em}
\hrule \vspace{.10in}}
\renewcommand{\part}[1] {\vspace{.10in} {\bf (#1)}}

\newcommand{\myname}{Chang-Hyun Mungai}
\newcommand{\myhwnum}{Theory and Specification Notes}

\setlength{\parindent}{0pt}
\setlength{\parskip}{5pt plus 1pt}
 
\pagestyle{fancyplain}
\lhead{\fancyplain{}{\textbf{\myhwnum}}}      % Note the different brackets!

\begin{document}

\medskip                        % Skip a "medium" amount of space
                                % (latex determines what medium is)
                                % Also try: \bigskip, \littleskip

\thispagestyle{plain}
\begin{center}                  % Center the following lines
{\Large Theory and Specification} \\
\end{center}

\question{Principles}

\begin{enumerate}
  \item It must be simple, object-oriented, and familiar.
  \item It must be robust and secure.
  \item It must be architecture-neutral and portable.
  \item It must execute with high performance.
  \item It must be interpreted, threaded, and dynamic.
\end{enumerate}

\question{Basic Definitions}

\begin{itemize}
  \item an object is a runtime entity and it’s state is stored in fields and behavior is shown via methods
  \begin{itemize}
    \item methods operate on an object's internal state and serve as the primary mechanism for object-to-object communication
    \item the Object class is the parent class of all the classes in java by default
  \end{itemize}
  \item a class represents the set of properties or methods that are common to all objects of one type
  \begin{itemize}
    \item a class can contain fields and methods to describe the behavior of an object
  \end{itemize}
    \item an interface is an abstract type that is used to specify a behavior that classes must implement
\end{itemize}

\question{Inheritance}

\begin{itemize}
  \item the class which inherits the properties of other is known as subclass (derived class, child class) 
  \item the class whose properties are inherited is known as superclass (base class, parent class).
  \item extends is the keyword used to inherit the properties of a class
\begin{lstlisting}
class Super {
   .....
   .....
}
class Sub extends Super {
   .....
   .....
}
\end{lstlisting}
\end{itemize}

\question{Access Modifiers}

\begin{itemize}
  \item public

  \begin{itemize}
    \item any class can access
    \item accessible by entire application
  \end{itemize}

  \item private

  \begin{itemize}
    \item only accessible within the class
  \end{itemize}

  \item protected

  \begin{itemize}
    \item allow the class itself to access them
    \item classes inside of the same package to access them
    \item subclasses of that class to access them
  \end{itemize}

  \item package protected

  \begin{itemize}
    \item default
    \item the same class and any class in the same package has access
    \item protected minus the subclass unless subclass is in package
  \end{itemize}

  \item Static: Belongs to class not an instance of the class
\end{itemize}

\question{Type Classifications}

\begin{itemize}
  \item Concrete Types
  \begin{itemize}
    \item concrete types describe object implementations, including memory layout and the code executed upon method invocation
    \item the exact class of which an object is an instance not the more general set of the class and its subclasses
    \item beware of falling into the trap of thinking that all concrete types are single classes!
    \item Set of Exact Classes
    \item ex: List x has concrete type {ArrayList, LinkedList, ...}
  \end{itemize}
  \item Abstract Types
  \begin{itemize}
    \item Abstract types, on the other hand, describe properties of objects
    \item They do not distinguish between different implementations of the same behavior
    \item Java provides abstract types in the form of interfaces, which list the fields and operations that implementations must support
  \end{itemize}
\end{itemize}

\question{Generics}

\begin{itemize}
  \item Definition
  \begin{itemize}
    \item generics are a facility of generic programming
    \begin{itemize}
       \item a style of computer programming in which algorithms are written in terms of types to-be-specified-later that are then instantiated when needed for specific types provided as parameters
    \end{itemize}
    \item ex: compiletime: 
    \begin{lstlisting}
    List<String> 
    \end{lstlisting}
    runtime: List
  \end{itemize}
  \item Notes
  \begin{itemize}
    \item in java, generics are only checked at compile time for type correctness
    \item generic type information is then removed via a process called type erasure, to maintain compatibility with old JVM implementations, making it unavailable at runtime
  \end{itemize}
  \item Sources
  \begin{itemize}
    \item \url{https://en.wikipedia.org/wiki/Generics_in_Java}
  \end{itemize}
\end{itemize}

\question{Sources}

\begin{itemize}
  \item \url{https://en.wikipedia.org/wiki/Java_(programming_language)}
  \item \url{https://www.tutorialspoint.com/java/java_interview_questions.htm}
\end{itemize}

\end{document}

