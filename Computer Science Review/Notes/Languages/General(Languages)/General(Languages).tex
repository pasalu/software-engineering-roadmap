% Start Header
\documentclass[11pt]{article}

\usepackage{amsmath}
\usepackage{amssymb}
\usepackage{fancyhdr}
\usepackage{listings}
\usepackage{color}
\usepackage{graphicx}
\graphicspath{ {images/} }
\usepackage{hyperref}
\usepackage{mathtools}

\newcommand\Myperm[2][n]{\prescript{#1\mkern-2.5mu}{}P_{#2}}
\newcommand\Mycomb[2][n]{\prescript{#1\mkern-0.5mu}{}C_{#2}}

\definecolor{dkgreen}{rgb}{0,0.6,0}
\definecolor{gray}{rgb}{0.5,0.5,0.5}
\definecolor{mauve}{rgb}{0.58,0,0.82}

\def\code#1{\texttt{#1}}

\lstset{frame=tb,
  language=Java,
  aboveskip=3mm,
  belowskip=3mm,
  showstringspaces=false,
  columns=flexible,
  basicstyle={\small\ttfamily},
  numbers=none,
  numberstyle=\tiny\color{gray},
  keywordstyle=\color{blue},
  commentstyle=\color{dkgreen},
  stringstyle=\color{mauve},
  breaklines=true,
  breakatwhitespace=true,
  tabsize=3
}

\oddsidemargin0cm
\topmargin-2cm     %I recommend adding these three lines to increase the 
\textwidth16.5cm   %amount of usable space on the page (and save trees)
\textheight23.5cm  

\newcommand{\question}[2] {\vspace{.25in} \hrule\vspace{0.5em}
\noindent{\bf #1: #2} \vspace{0.5em}
\hrule \vspace{.10in}}
\renewcommand{\part}[1] {\vspace{.10in} {\bf (#1)}}

\newcommand{\myname}{Chang-Hyun Mungai}

\setlength{\parindent}{0pt}
\setlength{\parskip}{5pt plus 1pt}
 
\pagestyle{fancyplain}
\lhead{\fancyplain{}{\textbf{\myhwnum}}}      % Note the different brackets!

% End Header

\newcommand{\myhwnum}{General(Languages) Notes}

\begin{document}

\medskip                        % Skip a "medium" amount of space
                                % (latex determines what medium is)
                                % Also try: \bigskip, \littleskip

\thispagestyle{plain}
\begin{center}                  % Center the following lines
{\Large General(Languages)} \\
\end{center}

\question{Classification}

\begin{itemize}
  \item Abstraction
  \begin{itemize}
    \item Declarative
    \begin{itemize}
      \item Functional
    \end{itemize}
    \item Imperative
    \begin{itemize}
      \item Procedural
    \end{itemize}
  \end{itemize}
  \item Behavior
  \begin{itemize}
    \item Dynamic
    \item Static
  \end{itemize}
\end{itemize}

\question{Declarative Languages}

\begin{itemize}
  \item not imperative
  \item describes what computation should be performed and not how to compute it
  \item lacks side effects (referentially transparent)
  \begin{itemize}
    \item an expression always evaluates to the same result in any context
    \item instance can be replaced with its corresponding value without changing the program's behavior
  \end{itemize}
  \item clear correspondence to mathematical logic
\end{itemize}

\question{Functional Languages}

\begin{itemize}
  \item Def
  \begin{itemize}
    \item define programs and subroutines as mathematical functions
    \item many functional languages are "impure", containing imperative features
    \item many functional languages are tied to mathematical calculation tools
    \item declarative: programming is done with expressions or declarations instead of statements
    \item the output value of a function depends only on the arguments that are input to the function
    \begin{itemize}
      \item calling a function f twice with the same value for an argument x will produce the same result f(x) each time
      \item eliminating side effects(changes in state that do not depend on the function inputs) makes it much easier to understand and predict the behavior of a program
    \end{itemize}
  \end{itemize}
  \item Pure ex: Haskell
  \item Impure ex: SML
\end{itemize}

\question{Imperative Languages}

\begin{itemize}
  \item Def
  \begin{itemize}
    \item uses statements that change a program's state
  \end{itemize}
\end{itemize}

\question{Scripting Languages}

\question{Hierarchy of programming languages}

\begin{itemize}
  \item High-Level language (C, Java, PHP, Python)
  \begin{itemize}
    \item more complex than machine code
  \end{itemize}
  \item Assembly language
  \begin{itemize}
    \item machine code with names substituted for numbers
  \end{itemize}
  \item Machine code
  \begin{itemize}
    \item only numbers
  \end{itemize}
  \item Hardware
  \item convert program into machine language
  \begin{itemize}
    \item compile the program
    \item interpret the program
  \end{itemize}
\end{itemize}

\question{Web Development Languages}

\begin{itemize}
  \item HTML/XHTML
  \begin{itemize}
    \item Defines content of web page
  \end{itemize}
  \item CSS
  \begin{itemize}
    \item appearance of web page
  \end{itemize}
  \item HTML+CSS can create static web pages
  \begin{itemize}
    \item static pages can interact with your visitor through the use of forms
    \item form is fill, request submitted and sent back to the server, new static web page is constructed and downloaded into the browser
    \item disadavantage of static web pages, only way visitor interact with the page is by filling out the form and waiting for a new page to load
  \end{itemize}
  \item javascript
  \begin{itemize}
    \item behavior of web page
    \item can validate each of the fields as visitor enter and provide immediate feedback when they make a typo (vs after they filled everything and submitted)
    \item add animations into the page which either attract attention to a specific part of the page or which make the page easier to use
    \item provide responses within the web page to various actions that your visitor takes
    \item load new images, objects, or scripts into the web page without needing to reload the entire page
    \item pass requests back to the server and handle responses from the server without the need for loading new pages
    \item not everyone visiting your page will have JavaScript and so your page will still need to work for those who don't have JavaScript
  \end{itemize}
\end{itemize}

\question{sources}

\begin{itemize}
  \item \url{https://en.wikipedia.org}
  \item \url{https://wiki.haskell.org/Referential_transparency}
\end{itemize}

\end{document}

