% You should title the file with a .tex extension (hw1.tex, for example)
\documentclass[11pt]{article}

\usepackage{amsmath}
\usepackage{amssymb}
\usepackage{fancyhdr}
\usepackage{listings}
\usepackage{color}
\usepackage{graphicx}
\graphicspath{ {images/} }
\usepackage{hyperref}
\usepackage{mathtools}

\definecolor{dkgreen}{rgb}{0,0.6,0}
\definecolor{gray}{rgb}{0.5,0.5,0.5}
\definecolor{mauve}{rgb}{0.58,0,0.82}

\lstset{frame=tb,
  language=Java,
  aboveskip=3mm,
  belowskip=3mm,
  showstringspaces=false,
  columns=flexible,
  basicstyle={\small\ttfamily},
  numbers=none,
  numberstyle=\tiny\color{gray},
  keywordstyle=\color{blue},
  commentstyle=\color{dkgreen},
  stringstyle=\color{mauve},
  breaklines=true,
  breakatwhitespace=true,
  tabsize=3
}

\oddsidemargin0cm
\topmargin-2cm     %I recommend adding these three lines to increase the 
\textwidth16.5cm   %amount of usable space on the page (and save trees)
\textheight23.5cm  

\newcommand{\question}[2] {\vspace{.25in} \hrule\vspace{0.5em}
\noindent{\bf #1: #2} \vspace{0.5em}
\hrule \vspace{.10in}}
\renewcommand{\part}[1] {\vspace{.10in} {\bf (#1)}}

\newcommand{\myname}{Chang-Hyun Mungai}
\newcommand{\myhwnum}{Recursive/Iterative}

\setlength{\parindent}{0pt}
\setlength{\parskip}{5pt plus 1pt}
 
\pagestyle{fancyplain}
\lhead{\fancyplain{}{\textbf{\myhwnum}}}      % Note the different brackets!

\begin{document}

\medskip                        % Skip a "medium" amount of space
                                % (latex determines what medium is)
                                % Also try: \bigskip, \littleskip

\thispagestyle{plain}
\begin{center}                  % Center the following lines
{\Large Recursive/Iterative} \\
\end{center}

\question{Definitions}

\begin{itemize}
  \item Recursive

  \begin{itemize}
    \item a function calls itself again and again till the base condition(stopping condition) is satisfied
    \item common to functional programming
  \end{itemize}

  \item Iterative

  \begin{itemize}
    \item iterative is used to describe a situation in which a sequence of instructions can be executed multiple times
    \item each time an iteration
  \end{itemize}

\end{itemize}

\question{Recursive Rules}

\begin{itemize}
  \item each recursive call should be on a smaller instance of the same problem, that is, a smaller subproblem
  \item recursive calls must eventually reach a base case, which is solved without further recursion

  \begin{itemize}
    \item Base case
    \item Recursive Case
  \end{itemize}

  \item Dynamic Programming and recursion

  \begin{itemize}
    \item use when ever you compute a recursive input multiple times
    \item memoization (caching previosly computed results, use result next time computation is needed)
  \end{itemize}

\end{itemize}

\question{Example problems}

\begin{itemize}
  \item drawing fractals
  \item factorial
  \item towers of hanoi
\end{itemize}

\question{Comparison}

\begin{itemize}
  \item factorial

  \begin{itemize}
    \item Iterative
\begin{lstlisting}
def factorial(n):
    factorial = 1
    for i in range(2,n+1):
        factorial *= i
    return factorial
\end{lstlisting}
    \item Recursive
\begin{lstlisting}
def factorial(n):
    if (n < 2):
        return 1
    else:
        return n*factorial(n-1)
\end{lstlisting}
  \end{itemize}

  \item Generally recursion more elegant, iteration better performance and debugability 
  \item Iterative algorithm

  \begin{itemize}
    \item more lines of code
    \item faster
  \end{itemize}

  \item Recursive algorithm

  \begin{itemize}
    \item complex to implement
    \item code will be elegant and easy to read
    \item tracing is difficult
    \item takes more time because of overheads like calling functions and registering stacks repeatedly
    \item some complex problems can be solved easily and effectively in recursion
  \end{itemize}

\end{itemize}

\end{document}

