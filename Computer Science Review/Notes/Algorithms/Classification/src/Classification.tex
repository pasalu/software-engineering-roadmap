% You should title the file with a .tex extension (hw1.tex, for example)
\documentclass[11pt]{article}

\usepackage{amsmath}
\usepackage{amssymb}
\usepackage{fancyhdr}
\usepackage{listings}
\usepackage{color}
\usepackage{graphicx}
\graphicspath{ {images/} }
\usepackage{hyperref}
\usepackage{mathtools}

\definecolor{dkgreen}{rgb}{0,0.6,0}
\definecolor{gray}{rgb}{0.5,0.5,0.5}
\definecolor{mauve}{rgb}{0.58,0,0.82}

\lstset{frame=tb,
  language=Java,
  aboveskip=3mm,
  belowskip=3mm,
  showstringspaces=false,
  columns=flexible,
  basicstyle={\small\ttfamily},
  numbers=none,
  numberstyle=\tiny\color{gray},
  keywordstyle=\color{blue},
  commentstyle=\color{dkgreen},
  stringstyle=\color{mauve},
  breaklines=true,
  breakatwhitespace=true,
  tabsize=3
}

\oddsidemargin0cm
\topmargin-2cm     %I recommend adding these three lines to increase the 
\textwidth16.5cm   %amount of usable space on the page (and save trees)
\textheight23.5cm  

\newcommand{\question}[2] {\vspace{.25in} \hrule\vspace{0.5em}
\noindent{\bf #1: #2} \vspace{0.5em}
\hrule \vspace{.10in}}
\renewcommand{\part}[1] {\vspace{.10in} {\bf (#1)}}

\newcommand{\myname}{Chang-Hyun Mungai}
\newcommand{\myhwnum}{Classification Notes}

\setlength{\parindent}{0pt}
\setlength{\parskip}{5pt plus 1pt}
 
\pagestyle{fancyplain}
\lhead{\fancyplain{}{\textbf{\myhwnum}}}      % Note the different brackets!

\begin{document}

\medskip                        % Skip a "medium" amount of space
                                % (latex determines what medium is)
                                % Also try: \bigskip, \littleskip

\thispagestyle{plain}
\begin{center}                  % Center the following lines
{\Large Classification} \\
\end{center}

\question{Classification by purpose}

\begin{itemize}
  \item each algorithm has a goal
  \item kind of purposes
  \begin{itemize}
    \item Sorting a list
  \end{itemize}
\end{itemize}

\question{Classification by implementation}

\begin{itemize}
  \item recursive or iterative
  \begin{itemize}
    \item a recursive algorithm calls itself repeatedly until a certain condition matches
    \item a iterative algorithm uses looping statements such as for loop, while loop or do-while loop
    \item every recursive version has an iterative equivalent iterative, and vice versa
  \end{itemize}
  \item logical or procedural
  \begin{itemize}
    \item an algorithm may be viewed as controlled logical deduction
    \item a logic component expresses the axioms which may be used in the computation 
    \item a control component determines the way in which deduction is applied to the axioms
  \end{itemize}
  \item serial or parallel
  \begin{itemize}
    \item in serial algorithms, computers execute one instruction of an algorithm at a time
    \item parallel algorithms take advantage of computer architectures to process several instructions at once
  \end{itemize}
  \item deterministic or non-deterministic
  \begin{itemize}
    \item deterministic algorithms solve the problem with a predefined process 
    \item non-deterministic algorithm must perform guesses of best solution at each step through the use of heuristics
  \end{itemize}
\end{itemize}

\question{Classification by design paradigm}

\begin{itemize}
  \item divide and conquer
  \item contraction (Reduction/transform and conquer)
  \item dynamic programming
  \item greedy method
  \begin{itemize}
    \item similar to dynamic programming but solutions to subproblems do not have to be known at each stage
    \item a "greedy" choice can be made of what looks the best solution for the moment
    \item Kruskal
  \end{itemize}
  \item linear programming
  \item graphs
  \item the probabilistic and heuristic paradigm
  \begin{itemize}
    \item probabilistic
    \begin{itemize}
      \item those that make some choices randomly
    \end{itemize}
    \item genetic
    \begin{itemize}
      \item attempt to find solutions to problems by mimicking biological evolutionary processes
      \item a cycle of random mutations yielding successive generations of "solutions"
      \item thus, they emulate reproduction and "survival of the fittest"
    \end{itemize}
    \item heuristic
    \begin{itemize}
      \item whose general purpose is not to find an optimal solution, but an approximate solution where the time or resources to find a perfect solution are not practical
    \end{itemize}
  \end{itemize}
\end{itemize}

\question{Sources}

\begin{itemize}
  \item \url{https://www.quora.com/Which-are-the-10-algorithms-every-computer-science-student-must-implement-at-least-once-in-life}
\end{itemize}

\end{document}

