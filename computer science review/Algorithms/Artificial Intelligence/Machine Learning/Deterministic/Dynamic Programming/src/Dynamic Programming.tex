% You should title the file with a .tex extension (hw1.tex, for example)
\documentclass[11pt]{article}

\usepackage{amsmath}
\usepackage{amssymb}
\usepackage{fancyhdr}
\usepackage{listings}
\usepackage{color}
\usepackage{graphicx}
\graphicspath{ {images/} }
\usepackage{hyperref}
\usepackage{mathtools}

\definecolor{dkgreen}{rgb}{0,0.6,0}
\definecolor{gray}{rgb}{0.5,0.5,0.5}
\definecolor{mauve}{rgb}{0.58,0,0.82}

\lstset{frame=tb,
  language=Java,
  aboveskip=3mm,
  belowskip=3mm,
  showstringspaces=false,
  columns=flexible,
  basicstyle={\small\ttfamily},
  numbers=none,
  numberstyle=\tiny\color{gray},
  keywordstyle=\color{blue},
  commentstyle=\color{dkgreen},
  stringstyle=\color{mauve},
  breaklines=true,
  breakatwhitespace=true,
  tabsize=3
}

\oddsidemargin0cm
\topmargin-2cm     %I recommend adding these three lines to increase the 
\textwidth16.5cm   %amount of usable space on the page (and save trees)
\textheight23.5cm  

\newcommand{\question}[2] {\vspace{.25in} \hrule\vspace{0.5em}
\noindent{\bf #1: #2} \vspace{0.5em}
\hrule \vspace{.10in}}
\renewcommand{\part}[1] {\vspace{.10in} {\bf (#1)}}

\newcommand{\myname}{Chang-Hyun Mungai}
\newcommand{\myhwnum}{Dynamic Programming Notes}

\setlength{\parindent}{0pt}
\setlength{\parskip}{5pt plus 1pt}
 
\pagestyle{fancyplain}
\lhead{\fancyplain{}{\textbf{\myhwnum}}}      % Note the different brackets!

\begin{document}

\medskip                        % Skip a "medium" amount of space
                                % (latex determines what medium is)
                                % Also try: \bigskip, \littleskip

\thispagestyle{plain}
\begin{center}                  % Center the following lines
{\Large Dynamic Programming} \\
\end{center}

\question{Dynamic Programming}

 Dynamic programming is a technique for solving problems recursively and is applicable when the computations of the subproblems overlap

\question{DP Tools}

\begin{itemize}
   \item Memoization (Top down)

  \begin{itemize}
     \item an optimization technique where you cache previously computed results, and return the cached result when the same computation is needed again
     \item storing the results of expensive function calls and returning the result when the same inputs occur again
  \end{itemize}

   \item Tabulation (Bottom Up)

  \begin{itemize}
     \item using iterative approach to solve the problem by solving the smaller sub- problems first and then using it during the execution of bigger problem
  \end{itemize}

   \item Comparison

  \begin{itemize}
     \item memoization usually requires more code and is less straightforward, but has computational advantages in some problems

    \begin{itemize}
       \item mainly those which you do not need to compute all the values for the whole matrix to reach the answer
    \end{itemize}

     \item tabulation is more straightforward, but may compute unnecessary values

    \begin{itemize}
       \item if you do need to compute all the values, this method is usually faster, though, because of the smaller overhead
    \end{itemize}

  \end{itemize}

\end{itemize}

\question{Examples}

\begin{itemize}
   \item Longest Common Subsequence problem
   \item Knapsack
   \item Travelling salesman problem
\end{itemize}

\question{Sources}

\begin{itemize}
  \item \url{http://stackoverflow.com/questions/12042356/memoization-or-tabulation-approach-for-dynamic-programming}
\end{itemize}

\end{document}

