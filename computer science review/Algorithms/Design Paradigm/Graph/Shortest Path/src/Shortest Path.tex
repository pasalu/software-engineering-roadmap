% You should title the file with a .tex extension (hw1.tex, for example)
\documentclass[11pt]{article}

\usepackage{amsmath}
\usepackage{amssymb}
\usepackage{fancyhdr}
\usepackage{listings}
\usepackage{color}
\usepackage{graphicx}
\graphicspath{ {images/} }
\usepackage{hyperref}
\usepackage{mathtools}

\definecolor{dkgreen}{rgb}{0,0.6,0}
\definecolor{gray}{rgb}{0.5,0.5,0.5}
\definecolor{mauve}{rgb}{0.58,0,0.82}

\lstset{frame=tb,
  language=Java,
  aboveskip=3mm,
  belowskip=3mm,
  showstringspaces=false,
  columns=flexible,
  basicstyle={\small\ttfamily},
  numbers=none,
  numberstyle=\tiny\color{gray},
  keywordstyle=\color{blue},
  commentstyle=\color{dkgreen},
  stringstyle=\color{mauve},
  breaklines=true,
  breakatwhitespace=true,
  tabsize=3
}

\oddsidemargin0cm
\topmargin-2cm     %I recommend adding these three lines to increase the 
\textwidth16.5cm   %amount of usable space on the page (and save trees)
\textheight23.5cm  

\newcommand{\question}[2] {\vspace{.25in} \hrule\vspace{0.5em}
\noindent{\bf #1: #2} \vspace{0.5em}
\hrule \vspace{.10in}}
\renewcommand{\part}[1] {\vspace{.10in} {\bf (#1)}}

\newcommand{\myname}{Chang-Hyun Mungai}
\newcommand{\myhwnum}{Shortest Path Notes}

\setlength{\parindent}{0pt}
\setlength{\parskip}{5pt plus 1pt}
 
\pagestyle{fancyplain}
\lhead{\fancyplain{}{\textbf{\myhwnum}}}      % Note the different brackets!

\begin{document}

\medskip                        % Skip a "medium" amount of space
                                % (latex determines what medium is)
                                % Also try: \bigskip, \littleskip

\thispagestyle{plain}
\begin{center}                  % Center the following lines
{\Large Shortest Path} \\
\end{center}

\question{Dijkstras}

Implementation with PQ

\begin{itemize}
\item O($\mid$E$\mid$ + $\mid$V$\mid$ log $\mid$V$\mid$)
\item Make source current node with its distance 0
\item Repeat until no elements in PQ
\begin{itemize}
\item If current node is not in distance map or has a greater value than computed value place the current node in the distance map (mapped to distance)
\item Place its neighbors in the PQ with their distances to current node + current node distance
\item Dequeue min element from PQ and make current node
\end{itemize}
\end{itemize}

\question{Bellman Ford}

psuedo java code

\begin{lstlisting}
for i=1 to size(vertices)-1
	for each edge (u,v)
		if distance[u]+w < distance[v]
			distance[v]=distance[u]+w
			predecessor[v]=u
\end{lstlisting}

c++ code

\begin{lstlisting}
function BellmanFord(list vertices, list edges, vertex source)
   ::distance[],predecessor[]

   // This implementation takes in a graph, represented as
   // lists of vertices and edges, and fills two arrays
   // (distance and predecessor) with shortest-path
   // (less cost/distance/metric) information

   // Step 1: initialize graph
   for each vertex v in vertices:
       if v is source then distance[v] := 0
       else distance[v] := inf
       predecessor[v] := null

   // Step 2: relax edges repeatedly
   for i from 1 to size(vertices)-1:
       for each edge (u, v) in Graph with weight w in edges:
           if distance[u] + w < distance[v]:
               distance[v] := distance[u] + w
               predecessor[v] := u

   // Step 3: check for negative-weight cycles
   for each edge (u, v) in Graph with weight w in edges:
       if distance[u] + w < distance[v]:
           error "Graph contains a negative-weight cycle"
   return distance[], predecessor[]
\end{lstlisting}

Notes

\begin{itemize}
\item O($\mid$V$\mid$ $\mid$E$\mid$)
\item Allows negative weights
\end{itemize}

\end{document}

