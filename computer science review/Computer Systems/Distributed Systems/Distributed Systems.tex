% Start Header
\documentclass[11pt]{article}

\usepackage{amsmath}
\usepackage{amssymb}
\usepackage{fancyhdr}
\usepackage{listings}
\usepackage{color}
\usepackage{graphicx}
\graphicspath{ {images/} }
\usepackage{hyperref}
\usepackage{mathtools}

\newcommand\Myperm[2][n]{\prescript{#1\mkern-2.5mu}{}P_{#2}}
\newcommand\Mycomb[2][n]{\prescript{#1\mkern-0.5mu}{}C_{#2}}

\definecolor{dkgreen}{rgb}{0,0.6,0}
\definecolor{gray}{rgb}{0.5,0.5,0.5}
\definecolor{mauve}{rgb}{0.58,0,0.82}

\def\code#1{\texttt{#1}}

\lstset{frame=tb,
  language=Java,
  aboveskip=3mm,
  belowskip=3mm,
  showstringspaces=false,
  columns=flexible,
  basicstyle={\small\ttfamily},
  numbers=none,
  numberstyle=\tiny\color{gray},
  keywordstyle=\color{blue},
  commentstyle=\color{dkgreen},
  stringstyle=\color{mauve},
  breaklines=true,
  breakatwhitespace=true,
  tabsize=3
}

\oddsidemargin0cm
\topmargin-2cm     %I recommend adding these three lines to increase the 
\textwidth16.5cm   %amount of usable space on the page (and save trees)
\textheight23.5cm  

\newcommand{\question}[2] {\vspace{.25in} \hrule\vspace{0.5em}
\noindent{\bf #1: #2} \vspace{0.5em}
\hrule \vspace{.10in}}
\renewcommand{\part}[1] {\vspace{.10in} {\bf (#1)}}

\newcommand{\myname}{Chang-Hyun Mungai}

\setlength{\parindent}{0pt}
\setlength{\parskip}{5pt plus 1pt}
 
\pagestyle{fancyplain}
\lhead{\fancyplain{}{\textbf{\myhwnum}}}      % Note the different brackets!

% End Header

\newcommand{\myhwnum}{Distributed Systems Notes}

\begin{document}

\medskip                        % Skip a "medium" amount of space
                                % (latex determines what medium is)
                                % Also try: \bigskip, \littleskip

\thispagestyle{plain}
\begin{center}                  % Center the following lines
{\Large Distributed Systems}
\end{center}

\question{General}

\begin{itemize}
  \item a system with multiple components located on different machines that communicate and coordinate actions in order to appear as a single coherent system to the end-user
  \item organizing resources via network with more latency, less bandwidth, and higher error rate
  \item compared to parallel systems
  \begin{itemize}
    \item Distributed has multiple (distributed users) vs parallel designed for single user or user process
    \item Parallel has less security issues
    \item Distributed Systems: cooperative work environment, Parallel Systems: environment designed to provide the maximum parallelization and speed-up for a single task
  \end{itemize}
\end{itemize}

\question{Definitions}

\begin{itemize}
  \item task: collection of resources configured to solve a particular problem
  \begin{itemize}
    \item contains not only the open files and communication channels, but also the threads (a.k.a. processes)
    \item a task is a factory, all of the means of production scattered across many assembly lines
  \end{itemize}
\end{itemize}

\question{Migrating}

\begin{itemize}
  \item Migrating Computation
  \begin{itemize}
    \item need to move resources (memory, files)
    \item move state of interaction with resources
    \item need to restore communication (other side need to know new location)
  \end{itemize}
  \item Migrating File State
  \begin{itemize}
    \item keep track of the essential file state within the process and be ready to recreate it
      \item migration system guarantees that higher level file operations are atomic, operation opens the file, seeks as needed, performs a read or write, and then closes the file
      \item maintain the same file state, but only reopen the file when needed
  \end{itemize}
  \item Migrating Communication Sessions
  \begin{itemize}
    \item need to reestablish and map global state, such as network sockets
    \item higher level network name
    \item edge cases, such as what happens when a process move mid-transmission
  \end{itemize}
\end{itemize}

\question{Networking}

\begin{itemize}
  \item LAN:  Local Area Network a homogenous network
  \begin{itemize}
    \item to communicate broadcast (yell) and all stations hear the broadcast
    \item station have station id or LAN Address and messages have sender and receiver id
    \begin{itemize}
      \item everyone can hear other stations messages but ignore it unless diagnostic or malicious
    \end{itemize}
    \item size is self-limiting
    \begin{itemize}
      \item longer wire the weaker the signal
      \item greater the distance through the air the weaker the signal
      \item network can be clogged with collision, can collapse with utilization as low as 30%
    \end{itemize}
  \end{itemize}
  \item bridge
  \begin{itemize}
    \item connect LANs
    \item send message for station only to relevant LAN (hashtable)
    \item bridge managed so if station changes LAN
    \item bridges (through configuration) create a spanning tree of location to prevent cycles
    \begin{itemize}
      \item if bridge fails form a different tree to get around failure
    \end{itemize}
    \item can't create bridges for the whole planet
  \end{itemize}
  \item IP Address
  \begin{itemize}
    \item IP4
    \begin{itemize}
      \item Class A (huge): 8 bits(network) + 24 bits (host) address begin with 0
      \item Class B (big): 16 bits(network) + 16 bits (host) address begin with 10
      \item Class C (small): 24 bits(network) + 8 bits (host) address begin with 110
    \end{itemize}
    \item IP6
    \begin{itemize}
      \item Classless, first few bits to describe the network/host division
      \item 73.93.0.0/15
      \begin{itemize}
        \item address with 15 network bits and 17 host bits
      \end{itemize}
    \end{itemize}
  \end{itemize}
\end{itemize}

\question{sources}

\begin{itemize}
\item \url{http://www.andrew.cmu.edu/course/15-440-f14/index/lecture_index.html}
\item \url{https://blog.stackpath.com/distributed-system/}
\end{itemize}

\end{document}

