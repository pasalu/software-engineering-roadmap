% You should title the file with a .tex extension (hw1.tex, for example)
\documentclass[11pt]{article}

\usepackage{amsmath}
\usepackage{amssymb}
\usepackage{fancyhdr}
\usepackage{listings}
\usepackage{color}
\usepackage{graphicx}
\graphicspath{ {images/} }
\usepackage{hyperref}
\usepackage{mathtools}

\definecolor{dkgreen}{rgb}{0,0.6,0}
\definecolor{gray}{rgb}{0.5,0.5,0.5}
\definecolor{mauve}{rgb}{0.58,0,0.82}

\lstset{frame=tb,
  language=Java,
  aboveskip=3mm,
  belowskip=3mm,
  showstringspaces=false,
  columns=flexible,
  basicstyle={\small\ttfamily},
  numbers=none,
  numberstyle=\tiny\color{gray},
  keywordstyle=\color{blue},
  commentstyle=\color{dkgreen},
  stringstyle=\color{mauve},
  breaklines=true,
  breakatwhitespace=true,
  tabsize=3
}

\oddsidemargin0cm
\topmargin-2cm     %I recommend adding these three lines to increase the 
\textwidth16.5cm   %amount of usable space on the page (and save trees)
\textheight23.5cm  

\newcommand{\question}[2] {\vspace{.25in} \hrule\vspace{0.5em}
\noindent{\bf #1: #2} \vspace{0.5em}
\hrule \vspace{.10in}}
\renewcommand{\part}[1] {\vspace{.10in} {\bf (#1)}}

\newcommand{\myname}{Chang-Hyun Mungai}
\newcommand{\myhwnum}{Compiler notes}

\setlength{\parindent}{0pt}
\setlength{\parskip}{5pt plus 1pt}
 
\pagestyle{fancyplain}
\lhead{\fancyplain{}{\textbf{\myhwnum}}}      % Note the different brackets!

\begin{document}

\medskip                        % Skip a "medium" amount of space
                                % (latex determines what medium is)
                                % Also try: \bigskip, \littleskip

\thispagestyle{plain}
\begin{center}                  % Center the following lines
{\Large Compiler} \\
\end{center}

\question{Definition}

a computer program that translates computer code written in one programming language (the source language) into another language (the target language)

\question{Compilation Types (C)}

\begin{itemize}
  \item hello.c (source program, text)
  \item hello.i (modified source program, text) preprocessor (cpp)
  \begin{itemize}
    \item modifies according to $\#$ directives ($\#$include$\textless$stdio.h$\textgreater$)
  \end{itemize}
  \item hello.s (assembly program, text) compiler (cc1)
  \begin{itemize}
    \item assembly  \item language program (statement equals 1 low level machine language instruction)
    \item assembly language common compiler output for different high level languages
  \end{itemize}
  \item hello.o (relocatable object programs, binary) assembler (as)
  \begin{itemize}
    \item to binary file (relocatable object program) whose bytes encode machine language instructions
  \end{itemize}
  \item hello (executable object program, binary) linker (ld) (linked with printf.o)
  \begin{itemize}
    \item linker merges standard c library functions (executable object file)
  \end{itemize}
\end{itemize}

\question{Sources}

\begin{itemize}
\item Computer Systems: A Programmers Perspective pg 4-7
\end{itemize}

\end{document}

