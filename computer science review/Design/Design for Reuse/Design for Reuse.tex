% You should title the file with a .tex extension (hw1.tex, for example)
\documentclass[11pt]{article}

\usepackage{amsmath}
\usepackage{amssymb}
\usepackage{fancyhdr}
\usepackage{listings}
\usepackage{color}
\usepackage{graphicx}
\graphicspath{ {images/} }
\usepackage{hyperref}
\usepackage{mathtools}

\definecolor{dkgreen}{rgb}{0,0.6,0}
\definecolor{gray}{rgb}{0.5,0.5,0.5}
\definecolor{mauve}{rgb}{0.58,0,0.82}

\lstset{frame=tb,
  language=Java,
  aboveskip=3mm,
  belowskip=3mm,
  showstringspaces=false,
  columns=flexible,
  basicstyle={\small\ttfamily},
  numbers=none,
  numberstyle=\tiny\color{gray},
  keywordstyle=\color{blue},
  commentstyle=\color{dkgreen},
  stringstyle=\color{mauve},
  breaklines=true,
  breakatwhitespace=true,
  tabsize=3
}

\oddsidemargin0cm
\topmargin-2cm     %I recommend adding these three lines to increase the 
\textwidth16.5cm   %amount of usable space on the page (and save trees)
\textheight23.5cm  

\newcommand{\question}[2] {\vspace{.25in} \hrule\vspace{0.5em}
\noindent{\bf #1: #2} \vspace{0.5em}
\hrule \vspace{.10in}}
\renewcommand{\part}[1] {\vspace{.10in} {\bf (#1)}}

\newcommand{\myname}{Chang-Hyun Mungai}
\newcommand{\myhwnum}{Design for Reuse notes}

\setlength{\parindent}{0pt}
\setlength{\parskip}{5pt plus 1pt}
 
\pagestyle{fancyplain}
\lhead{\fancyplain{}{\textbf{\myhwnum}}}      % Note the different brackets!

\begin{document}

\medskip                        % Skip a "medium" amount of space
                                % (latex determines what medium is)
                                % Also try: \bigskip, \littleskip

\thispagestyle{plain}
\begin{center}                  % Center the following lines
{\Large Design for Reuse} \\
\end{center}

\question{Delegation}

\begin{itemize}
  \item Delegation is	simply when one	object relies on another object for some subset	of its functonality
\begin{itemize}
  \item Sorter is delegating functonality to	some Comparator implementation
\end{itemize}
  \item Judicious delegation enables code reuse
\begin{itemize}
  \item Sorter can be reused with arbitrary sort orders
  \item Comparators can be reused with arbitrary client code that needs to compare integers
\end{itemize}
\end{itemize}

\question{Delegation and design}

\begin{itemize}
  \item Small interfaces
  \item Classes to encapsulate algorithms
\begin{itemize}
  \item ex: the Comparator, the Strategy pattern
\end{itemize}
\end{itemize}

\question{Inheritance}

\begin{itemize}
  \item Typical roles:
\begin{itemize}
  \item An interface defines	expectations/commitment for clients
  \item An abstract class is a convenient hybrid between an interface and a full implementation
  \item A subclass overrides a method definition to specialize its implementation
\end{itemize}
\end{itemize}

\question{Benefits of Inheritance}

\begin{itemize}
  \item Reuse	of code
  \item Modeling flexibility
  \item A Java aside:
\begin{itemize}
  \item Each class can directly extend only one parent class
  \item A class can implement multiple interfaces
\end{itemize}
\end{itemize}

\question{Power of object oriented interfaces}

\begin{itemize}
  \item Subtype polymorphism
\begin{itemize}
  \item Different kinds of objects can be treated	uniformly by client code
\end{itemize}
  \item e.g.,	a list of all accounts
\begin{itemize}
  \item Each object behaves according to its type
\end{itemize}
  \item If you add new kind of account, client code does not change
\end{itemize}

\question{Inheritance and subtyping}

\begin{itemize}
  \item Inheritance is for code reuse
\begin{itemize}
  \item Write	code once and only once
  \item Superclass features implicitly available in subclass
\end{itemize}
  \item Subtyping is for polymorphism
\begin{itemize}
  \item Accessing objects the same way, but getting different behavior	
  \item Subtype is substitutable for supertype
\end{itemize}
\end{itemize}

\question{Java details: final}

\begin{itemize}
  \item A final field: prevents reassignment to the field after initialization
  \item A final method: prevents overriding the method
  \item A final class: prevents extending the class
\begin{itemize}
  \item e.g., public final class CheckingAccountImpl
\end{itemize}
\end{itemize}

\question{Type Casting}

\begin{itemize}
  \item Sometimes you wan a different type than what you have
\begin{itemize}
  \item ex: float pi=3.14; int indianapi=(int) pi;
\end{itemize}
  \item Useful if you have more specific subtype
  \item Advice: avoid downcasting types
\begin{itemize}
  \item Never downcast within superclass to a subclass
\end{itemize}
\end{itemize}

\question{Behavioral Subtyping}

\begin{itemize}
  \item Compiler enforced rules in Java:
\begin{itemize}
  \item Subtype (subclass, subinterface, object implementing interface) can add but not remove methods
  \item Overriding method must return same type or subtype
  \item Overriding method must accept same parameter types
  \item Overriding method may not throw additional exceptions
  \item Concrete class must  implement all undefined interface methods and abstract methods
\end{itemize}
  \item A subclass must fulfill all contracts its superclass does:
\begin{itemize}
  \item same or strong invariants
  \item same or stronger post  \item conds for all methods
  \item same or weaker pre  \item conds for all methods
\end{itemize}
\end{itemize}

\question{Parametric polymorphism via java generics}

\begin{itemize}
  \item Parametric polumorphism is the ability to define a type generically to allow static type  \item checking with out fully specifying types
  \item The java.util.Stack instead
\begin{itemize}
  \item A stack of some type T:
\begin{lstlisting}
public class Stack<T> {
	public void push(T obj){...}
	public T pop() { ...}
}
\end{lstlisting}
\end{itemize}
  \item Improves typechecking, simplifies client code
\end{itemize}

\question{Notes}

\begin{itemize}
  \item Template method design pattern
  \item Decorator design pattern
\end{itemize}

\end{document}

